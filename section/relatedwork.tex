\documentclass[../thesis.tex]{subfiles}
\begin{document}
\chapter{Related Work}
\label{ch:relatedwork}
Developers have created various packages to provide simulation of regulatory systems to aid researchers. The current packages, however, all have certain limitations and fail to fully satisfy the demands of researchers. In this section, I will review the advantages and disadvantages of existing programs and packages.

\section{COPASI}
	COPASI, for example, is one of the most widely used CPU based software for analyzing biological regulatory networks \cite{hoops2006copasi}. COPASI includes user-friendly interfaces for model input to cater to all researchers with varied programming ability. Simulation and analysis sections of COPASI perform efficiently on small-scale models, namely, one-cell models. However, the features included in the software strictly limit a pre-compiled program like COPASI and thus it is impossible for researchers to extend its function to multi-cell simulation of biological networks. For the same reason, researchers are not able to analyze complex system involving DDEs because COPASI only supports simulations with ordinary differential equations. 
\section{GPU Accelerated Packages}
On the other hand, simulation software packages that utilize GPU acceleration have shown considerable speedups. For example, cupSODA achieved``a 86x speedup on GPUs with respect to equivalent executions of LSODA on the CPU'' and Murray's package achieved ``speedups of up to 115-fold over comparable serial CPU implementations, and 15-fold over multithreaded CPU code'' \cite{nobile2012gpu,hoang2013novel,murray2012gpu}. However, very few of the existing packages support partial differential equations and none of them support DDE. Lack of functionality in these systems also exclude a large portion of researchers from using this category of packages. In addition, most of the existing packages are presented as libraries and are not nearly as user-friendly as COPASI; often written in languages like C/C++, those packages require even higher levels of programming skill, which is not common among researchers in this field. Until now, there is virtually no usage of such packages in this field.
\section{Other Software}
	Because of the aforementioned limitations of the packages described above, researchers interested in creating large and complex models are forced to create unique simulations. Professor Ay and I were tasked with building a specialized system. The system was constructed gradually over the past four years.  One of the major data structure in the system is a large three-dimensional array named $baby\_cl$, which holds concentration levels over several delayed time steps for all species and all cells. It supports DDE and multi-cell simulations, particularly for the segmentation clock project \cite{ay2013short,ay2014spatial}. There are a total of six mutants for each parameter set. The system uses thirteen differential equations to represent rate change at any time step. Historical data, stored in concentration levels, and differential equations together predict the concentration level of next time step. ODEs in the system only rely on the data in the last time step whereas DDEs in the system may request concentration levels from over one thousand time steps ago. There are some limitations to this system, however. Written for a CPU only environment, the system runs slowly and parameter estimation takes days to execute on Colgate's computer cluster. Furthermore, because all data structures and functions were hard coded for a particular regulatory network, updating the system corresponding to a model update was extremely inconvenient and highly time-consuming.
\end{document}
