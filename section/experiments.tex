\documentclass[../thesis.tex]{subfiles}
\begin{document}
\chapter{Validation}
\label{ch:Validation}
One aspect of the new system that is important to biological users is the validity of the system. Is the new system conducting all parts of simulation correctly? One way to test this is to see if the new system can replicate the simulation results of the original system. Controlled tests are conducted to check the accuracy of the simulation in this system. I passed the same set of simulation results into both the original system and the new system for simulation. After the concentration level of each species and each cell are recorded along the entire simulation, I compare the concentration levels from two systems to see if they are identical to each other. These controlled tests demonstrate a $0.3\%$ discrepancy in final simulation results after $60,000$ time steps (equivalent to $600$ minutes, a common length for the \textit{segmentation clock} project). \\
\\
 This slight discrepancy in simulation results may stem from the differences in implementation of the system. As described in the simulation section, the original system updates concentration level of a species based on the corresponding differential equation, which only counts the influences of a reaction on that particular species at a time. This process will be repeated for each of the species and if a reaction is related to multiple species, it will be included in multiple differential equations. On the other hand, the new system gathers all active rate changes that one reaction may impose on various species and then the system updates the concentration level for each of the species collectively. Potential rounding off error may arise since the order of two sub processes is switched. But overall, this is a relatively small error especially after $60,000$ time steps and will not affect any system level characteristics of the biological network.
\end{document}
